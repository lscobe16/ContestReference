\columnbreak
\section{Graphen}

\begin{algorithm}{Eulerkreis}
	\begin{methods}
		\method{euler}{berechnet den Kreis als Knotenarray}{\abs{V}+\abs{E}}
	\end{methods}
	\begin{itemize}
		\item Eulerkreis existiert, wenn jeder Knoten geraden Grad hat (ungerichtet),\\ bei jedem Knoten Ein- und Ausgangsgrad übereinstimmen (gerichtet).
		\item Eulerpfad existiert, wenn genau $\{0, 2\}$ Knoten ungeraden Grad haben (ungerichtet), bei allen Knoten Ein- und Ausgangsgrad übereinstimmen oder einer eine Ausgangskante mehr hat (Startknoten) und einer eine Eingangskante mehr hat (Endknoten).
		\item \textbf{Je nach Aufgabenstellung überprüfen, wie ein unzusammenhängender Graph interpretiert werden sollen.}
		\item Wenn eine bestimmte Sortierung verlangt wird oder Laufzeit vernachlässigbar ist, ist eine Implementierung mit einem \code{vector<set<int>> adj} leichter
		\item \textbf{Wichtig:} Algorithmus schlägt nicht fehl, falls kein Eulerzyklus existiert.
		Die Existenz muss separat geprüft werden.
	\end{itemize}
	\sourcecode{graph/euler.cpp}
\end{algorithm}

\begin{algorithm}{Lowest Common Ancestor \& Eulertour}
	\begin{methods}
		\method{LCA}{Vorberechnung: baut Eulertour über $G$ auf}{n\*\log(n)}
		\method{of}{findet LCA}{1}
		\method{dist}{Anzahl Kanten zwischen $u$ und $v$}{1}
	\end{methods}
	\sourcecode{graph/LCA_sparse.cpp}
\end{algorithm}

\begin{algorithm}{Link-Cut-Tree} % fits nicely onto one page
	\begin{methods}
		\method{Constructor}{baut Wald auf}{n}
		\method{connected}{prüft ob zwei Knoten im selben Baum liegen}{\log(n)}
		\method{link}{fügt $\{x,y\}$ Kante ein}{\log(n)}
		\method{cut}{entfernt $\{x,y\}$ Kante}{\log(n)}
		\method{lca}{berechnet LCA von $x$ und $y$}{\log(n)}
		\method{query}{berechnet \code{query} auf den Knoten des $xy$-Pfades}{\log(n)}
		\method{modify}{erhöht jeden wert auf dem $xy$-Pfad}{\log(n)}
	\end{methods}
	\sourcecode{datastructures/LCT.cpp}
\end{algorithm}

\begin{algorithm}{Centroids}
	\begin{methods}
		\method{find\_centroid}{findet alle Centroids des Baums (maximal 2)}{\abs{V}}
	\end{methods}
	\sourcecode{graph/centroid.cpp}
\end{algorithm}

\begin{algorithm}{Heavy-Light Decomposition}
	\begin{methods}
		\method{on\_path}{ruft $\O(\log{\abs{V}})$ Mal $f(l, rex)$ auf, sodass $u$-$v$-Pfad genau abgedeckt\\&
		Aufrufe unsortiert! Also ggf. per f extracten und dann sortieren}{}
		\method{on\_subtree}{ruft 1 Mal $f(l, rex)$ auf, sodass Subtree unter $u$ genau abgedeckt}{}
		\method{get\_lca}{= on\_path(., ., []\{\})}{}
	\end{methods}
	To initialize all vertices/edges in $\O(n)$ after construction, call on\_path for each vertex/edge.
	\sourcecode{graph/hld2.cpp}
\end{algorithm}

\begin{algorithm}{Baum-Isomorphie}
	\begin{methods}
		\method{treeLabel}{berechnet kanonischen Namen eines gewurzelten Baumes}{\abs{V}\*\log(\abs{V})}
	\end{methods}
	\sourcecode{graph/treeIsomorphism.cpp}
\end{algorithm}

\begin{algorithm}{Minimale Spannbäume}
	\paragraph{Schnitteigenschaft}
	Für jeden Schnitt $C$ im Graphen gilt:
	Gibt es eine Kante $e$, die echt leichter ist als alle anderen Schnittkanten, so gehört diese zu allen minimalen Spannbäumen.
	($\Rightarrow$ Die leichteste Kante in einem Schnitt kann in einem minimalen Spannbaum verwendet werden.)
	
	\paragraph{Kreiseigenschaft}
	Für jeden Kreis $K$ im Graphen gilt:
	Die schwerste Kante auf dem Kreis ist nicht Teil des minimalen Spannbaums.
\end{algorithm}

\columnbreak
\begin{algorithm}{Kruskal}
	\begin{methods}[ll]
		berechnet den Minimalen Spannbaum & \runtime{\abs{E}\cdot\log(\abs{E})} \\
	\end{methods}
	\sourcecode{graph/kruskal.cpp}
\end{algorithm}

\subsection{Kürzeste Wege}

\subsubsection{\textsc{Bellmann-Ford}-Algorithmus}
\method{bellmanFord}{kürzeste Pfade oder negative Kreise finden}{\abs{V}\*\abs{E}}
\sourcecode{graph/bellmannFord.cpp}

\subsubsection{Algorithmus von \textsc{Dijkstra}}
\method{dijkstra}{kürzeste Pfade in Graphen ohne negative Kanten}{\abs{E}\*\log(\abs{V})}
\sourcecode{graph/dijkstra.cpp}

\subsubsection{\textsc{Floyd-Warshall}-Algorithmus}
\method{floydWarshall}{kürzeste Pfade oder negative Kreise finden}{\abs{V}^3}
\begin{itemize}
	\item \code{dist[i][i] = 0, dist[i][j] = edge\{j, j\}.weight} oder \code{INF}
	\item \code{i} liegt auf einem negativen Kreis $\Leftrightarrow$ \code{dist[i][i] < 0}.
\end{itemize}
\sourcecode{graph/floydWarshall.cpp}

\subsubsection{Matrix-Algorithmus}
Sei $d_{i\smash{j}}$ die Distanzmatrix von $G$, dann gibt $d_{i\smash{j}}^k$ die kürzeste Distanz von $i$ nach $j$ mit maximal $k$ kanten an mit der Verknüpfung: $c_{i\smash{j}} = a_{i\smash{j}} \otimes b_{i\smash{j}} = \min\{a_{ik} b_{k\smash{j}}\}$


Sei $a_{ij}$ die Adjazenzmatrix von $G$ \textcolor{gray}{(mit $a_{ii} = 1$)}, dann gibt $a_{i\smash{j}}^k$ die Anzahl der Wege von $i$ nach $j$ mit Länge genau \textcolor{gray}{(maximal)} $k$ an mit der Verknüpfung: $c_{i\smash{j}} = a_{i\smash{j}} \otimes b_{i\smash{j}} = \sum a_{ik} \cdot b_{k\smash{j}}$


% \begin{algorithm}{Erd\H{o}s-Gallai}
% 	Sei $d_1 \geq \cdots \geq d_{n}$. Es existiert genau dann ein Graph $G$ mit Degreesequence $d$ falls $\sum\limits_{i=1}^{n} d_i$ gerade ist und für $1\leq k \leq n$: $\sum\limits_{i=1}^{k} d_i \leq k\cdot(k-1)+\sum\limits_{i=k+1}^{n} \min(d_i, k)$
% 	\begin{methods}
% 		\method{havelHakimi}{findet Graph}{(\abs{V}+\abs{E})\cdot\log(\abs{V})}
% 	\end{methods}
% 	\sourcecode{graph/havelHakimi.cpp}
% \end{algorithm}

\begin{algorithm}{Dynamic Connectivity}
	\begin{methods}
		\method{Constructor}{erzeugt Baum ($n$ Knoten, $m$ updates)}{n+m}
		\method{addEdge}{fügt Kannte ein,\code{id}=delete Zeitpunkt}{\log(n)}
		\method{eraseEdge}{entfernt Kante \code{id}}{\log(n)}
	\end{methods}
	\sourcecode{graph/connect.cpp}
\end{algorithm}

\begin{algorithm}{DFS}
	\input{graph/dfs}
\end{algorithm}

\begin{algorithm}{Artikulationspunkte, Brücken und BCC}
	\begin{methods}
		\method{find}{berechnet Artikulationspunkte, Brücken und BCC}{\abs{V}+\abs{E}}
	\end{methods}
	\textbf{Wichtig:} isolierte Knoten und Brücken sind keine BCC.
	\sourcecode{graph/articulationPoints.cpp}
\end{algorithm}

\begin{algorithm}{Strongly Connected Components (\textsc{Tarjan})}
	\label{scc}
	\begin{methods}
		\method{scc}{berechnet starke Zusammenhangskomponenten}{\abs{V}+\abs{E}}
	\end{methods}
	\sourcecode{graph/scc.cpp}
\end{algorithm}

\begin{algorithm}{2-SAT}
	\begin{methods}
		\method{solvable}{solves the 2-SAT, returns wether successful}{vars + clauses}
		& Ausgabe: \code{sccs}, {idx}
	\end{methods}
	\sourcecode{graph/2sat.cpp}
\end{algorithm}

\begin{algorithm}{Maximal Cliques}
	\begin{methods}
		\method{bronKerbosch}{berechnet alle maximalen Cliquen}{3^\frac{n}{3}}
		\method{addEdge}{fügt \textbf{ungerichtete} Kante ein}{1}
	\end{methods}
	\sourcecode{graph/bronKerbosch.cpp}
\end{algorithm}

\begin{algorithm}{Cycle Counting}
	\begin{methods}
		\method{findBase}{berechnet Basis}{\abs{V}\cdot\abs{E}}
		\method{count}{zählt Zykel}{2^{\abs{\mathit{base}}}}
	\end{methods}
	\begin{itemize}
		\item jeder Zyklus ist das xor von einträgen in \code{base}.
	\end{itemize}
	\sourcecode{graph/cycleCounting.cpp}
\end{algorithm}

\columnbreak
\begin{algorithm}{Wert des maximalen Matchings}
	Fehlerwahrscheinlichkeit: $\left(\frac{m}{MOD}\right)^I$\newline
	\begin{methods}
		\method{max\_matching}{berechnet (nur) den Wert des maximalen Matchings}{\abs{V}^3 \cdot I}
	\end{methods}
	\sourcecode{graph/matching.cpp}
\end{algorithm}

\begin{algorithm}{Allgemeines maximales Matching}
	\begin{methods}
		\method{match}{berechnet allgemeines Matching}{\abs{E}\*\abs{V}\*\log(\abs{V})}
	\end{methods}
	\sourcecode{graph/blossom.cpp}
\end{algorithm}

\columnbreak
\begin{algorithm}{Maximal Cardinatlity Bipartite Matching}
	\label{kuhn}
	\begin{methods}
		\method{kuhn}{berechnet Matching}{\abs{V}\*\min(ans^2, \abs{E})}
	\end{methods}
	\begin{itemize}
		\item die ersten [0..n) Knoten in \code{adj} sind die linke Seite des Graphen
		\item Wenn $|V| \approx |E|$, dann ist Dinic schneller ($n\sqrt{n}$ vs. $n^2$)
	\end{itemize}
	\sourcecode{graph/maxCarBiMatch.cpp}

	% TODO: wozu? wegkommentieren?
	\begin{methods}
		\method{hopcroft\_karp}{berechnet Matching}{\sqrt{\abs{V}}\*\abs{E}}
	\end{methods}
	\sourcecode{graph/hopcroftKarp.cpp}
\end{algorithm}

\begin{algorithm}{Maximum Weight Bipartite Matching}
	\begin{methods}
		\method{match}{berechnet Matching}{\abs{V}^3}
	\end{methods}
	\sourcecode{graph/maxWeightBipartiteMatching.cpp}
\end{algorithm}

\begin{algorithm}{Global Mincut}
	\begin{methods}
		\method{stoer\_wagner}{berechnet globalen Mincut}{\abs{V}^2\*\log(\abs{E})}
		\method{merge(a,b)}{merged Knoten $b$ in Knoten $a$}{\abs{E}}
	\end{methods}
	\textbf{Tipp:} Cut Rekonstruktion mit \code{unionFind} für Partitionierung oder \code{vector<bool>} für edge id's im cut.
	\sourcecode{graph/stoerWagner.cpp}
\end{algorithm}

\subsection{Max-Flow}
Für \textbf{ungerichtete} Kanten, einfach zwei gerichtete einfügen und beim auslesen zusammenfassen.
Für bipartites Matching nutze Kuhn (Seite \pageref{kuhn})

\optional{
\subsubsection{Capacity Scaling}
\begin{methods}
	\method{addEdge}{fügt eine \textbf{gerichtete} Kante ein}{1}
	\method{maxFlow}{gut bei dünn besetzten Graphen.}{\abs{E}^2\*\log(C)}
\end{methods}
\sourcecode{graph/capacityScaling.cpp}
}

\subsubsection{Push Relabel}
Gut für \textbf{sehr dicht besetzte Graphen}.\newline
% // TODO: nochmal anschauen
\begin{methods}
	\method{addEdge}{fügt eine \textbf{gerichtete} Kante ein}{1}
	\method{maxFlow}{berechnet Flow, siehe adj}{\abs{V}^2\*\sqrt{\abs{E}}}
\end{methods}
\sourcecode{graph/pushRelabel2.cpp}

\subsubsection{Dinic's Algorithm mit Capacity Scaling}
\begin{methods}
	\method{addEdge}{fügt eine \textbf{gerichtete} Kante ein}{1}
	\method{maxFlow}{berechnet Flow, siehe adj}{\abs{V}^2\cdot\abs{E}}
	\method{decompose}{zerlegt Flow in mehrere Pfade}{\abs{V}\cdot\abs{E}}
\end{methods}
\textbf{Unit-Graph} (jeder Knoten hat nur eine Ein-/Ausgehende Kante mit Kapazität 1)
$\Rightarrow$ $\O(\abs{E}\cdot\sqrt{\abs{V}})$.
Alle Kanten haben Kapazität 1 $\Rightarrow$ $\O(\abs{E}\cdot\sqrt{\abs{E}})$
\sourcecode{graph/dinicScaling.cpp}

\begin{algorithm}{Min-Cost-Max-Flow}
	\begin{methods}
		\method{addEdge}{fügt eine \textbf{gerichtete} Kante ein. Negative Kosten erlaubt.}{1} % TODO: negative Zyklen nicht erlaubt, für decomp?
		\method{run}{berechnet maximalen Fluss F mit minimalen Kosten}{\abs{V}\cdot\abs{E}\cdot F}
		& Chris: assume $\O(F (n+m) \log n)$
	\end{methods}
	\sourcecode{graph/minCostMaxFlow.cpp}
\end{algorithm}

\subsubsection{Anwendungen}
\begin{itemize}
	\item \textbf{Min-Cut}\newline
	Der maximale Fluss ist gleich dem minimalen Schnitt.
	\begin{enumerate}
		\item Bei Quelle $s$ und Senke $t$, partitioniere in $S$ und $T$.
		\item Zu $S$ gehören alle Knoten, die im Residualgraphen von $s$ aus erreichbar sind (Rückwärtskanten beachten).
	\end{enumerate}

	\item \textbf{Maximum Edge/Node Disjoint Paths}\newline
	Finde die maximale Anzahl Pfade von $s$ nach $t$, die keine Kante/Knoten teilen.
	\begin{enumerate}
		\item Setze $s$ als Quelle, $t$ als Senke und die Kapazität jeder Kante auf 1.
		\item Knoten: aufteilen in Eingangsknoten und Ausgangsknoten mit Cap-1-Kapazität.
		\item Der maximale Fluss entspricht den unterschiedlichen Pfaden ohne gemeinsame Kanten.
	\end{enumerate}

	\item \textbf{Minimum Flow}\newline
	Jede Edge hat minimum Flow (demands)
	\begin{itemize}
		\item $t \overset{\infty}\to s$, insert new $s'$ and $t'$, 
		minimum demands $u \overset{d}\to v$ get extracted by $s' \overset{+d}\to v$ and $u \overset{+d}\to t'$
		\item flow can be reconstructed with $flow + d$
		\item s' has to be saturated for flow to be valid
		\item Binary search $\infty$ to find minimum necessary flow
	\end{itemize}
	

	\item \textbf{Assignment Problem}
	NxN-Feld mit Werten auf jedem Feld. Finde minimale schwere Belegung, sodass in jeder Zeile und Spalte je genau ein Stein liegt.
	\begin{itemize}
		\item MinCostMaxFlow, vollständiger bipartiter Graph: links Zeilen, rechts Spalten, Cap 1, Cost übernehmen.
	\end{itemize}
\end{itemize}

\begin{algorithm}[optional]{TSP}
	\begin{methods}
		\method{TSP}{berechnet eine Tour}{n^2\*2^n}
	\end{methods}
	\sourcecode{graph/TSP.cpp}
\end{algorithm}

\begin{algorithm}[optional]{Bitonic TSP}
	\begin{methods}
		\method{bitonicTSP}{berechnet eine Bitonische Tour}{n^2}
	\end{methods}
	\sourcecode{graph/bitonicTSPsimple.cpp}
\end{algorithm}
